\documentclass[a4paper,12pt]{report}

%%% HarrixLaTeXDocumentTemplate
%%% Версия 1.22
%%% Шаблон документов в LaTeX на русском языке. Данный шаблон применяется в проектах HarrixTestFunctions, MathHarrixLibrary, Standard-Genetic-Algorithm  и др.
%%% https://github.com/Harrix/HarrixLaTeXDocumentTemplate
%%% Шаблон распространяется по лицензии Apache License, Version 2.0.

%%% Проверка используемого TeX-движка %%%
\usepackage{ifxetex}

%%% Поля и разметка страницы %%%
\usepackage{lscape} % Для включения альбомных страниц
\usepackage{geometry} % Для последующего задания полей

%%% Кодировки и шрифты %%%
\ifxetex
\usepackage{polyglossia} % Поддержка многоязычности
\usepackage{fontspec} % TrueType-шрифты
\else
\usepackage{cmap}  % Улучшенный поиск русских слов в полученном pdf-файле
\usepackage[T2A]{fontenc} % Поддержка русских букв
\usepackage[utf8]{inputenc} % Кодировка utf8
\usepackage[english, russian]{babel} % Языки: русский, английский
\usepackage{ textcomp }

\IfFileExists{pscyr.sty}{\usepackage{pscyr}}{} % Красивые русские шрифты
\fi

%%% Математические пакеты %%%
\usepackage{amsthm,amsfonts,amsmath,amssymb,amscd} % Математические дополнения от AMS
% Для жиного курсива в формулах %
\usepackage{bm}
% Для рисования некоторых математических символов (например, закрашенных треугольников)
\usepackage{mathabx}

%%% Оформление абзацев %%%
\usepackage{indentfirst} % Красная строка
\usepackage{setspace} % Расстояние между строками
\usepackage{enumitem} % Для список обнуление расстояния до абзаца

%%% Цвета %%%
\usepackage[usenames]{color}
\usepackage{color}
\usepackage{colortbl}

%%% Таблицы %%%
\usepackage{longtable} % Длинные таблицы
\usepackage{multirow,makecell,array} % Улучшенное форматирование таблиц

%%% Общее форматирование
\usepackage[singlelinecheck=off,center]{caption} % Многострочные подписи
\usepackage{soul} % Поддержка переносоустойчивых подчёркиваний и зачёркиваний
\usepackage{icomma} % Запятая в десятичных дробях

%%% Библиография %%%
\usepackage{cite}

%%% Гиперссылки %%%
\usepackage{hyperref}

%%% Изображения %%%
\usepackage{graphicx} % Подключаем пакет работы с графикой
\usepackage{epstopdf}
\usepackage{subcaption}

%%% Оглавление %%%
\usepackage{tocloft}

%%% Колонтитулы %%%
\usepackage{fancyhdr}

%%% Отображение кода %%%
\usepackage{xcolor}
\usepackage{listings}
\usepackage{caption}

%%% Псевдокоды %%%
\usepackage{algorithm} 
\usepackage{algpseudocode}

%%% Рисование графиков %%%
\usepackage{pgfplots}
 %Подключаем модуль пакетов
%%% HarrixLaTeXDocumentTemplate
%%% Версия 1.22
%%% Шаблон документов в LaTeX на русском языке. Данный шаблон применяется в проектах HarrixTestFunctions, MathHarrixLibrary, Standard-Genetic-Algorithm  и др.
%%% https://github.com/Harrix/HarrixLaTeXDocumentTemplate
%%% Шаблон распространяется по лицензии Apache License, Version 2.0.

%%% Макет страницы %%%
% Выставляем значения полей (ГОСТ 7.0.11-2011, 5.3.7)
\geometry{a4paper,top=2cm,bottom=2cm,left=2.5cm,right=1cm}

%%% Выравнивание и переносы %%%
\sloppy % Избавляемся от переполнений
\clubpenalty=10000 % Запрещаем разрыв страницы после первой строки абзаца
\widowpenalty=10000 % Запрещаем разрыв страницы после последней строки абзаца


%%% Библиография %%%
\makeatletter
\bibliographystyle{utf8gost71u}  % Оформляем библиографию по ГОСТ 7.1 (ГОСТ Р 7.0.11-2011, 5.6.7)
\renewcommand{\@biblabel}[1]{#1.} % Заменяем библиографию с квадратных скобок на точку
\makeatother

%%% Изображения %%%
\graphicspath{{images/}} % Пути к изображениям
% Поменять двоеточние на точку в подписях к рисунку
\RequirePackage{caption}
\DeclareCaptionLabelSeparator{defffis}{. }
\captionsetup{justification=centering,labelsep=defffis}

%%% Абзацы %%%
% Отсупы между строками
\singlespacing
\setlength{\parskip}{0.3cm} % отступы между абзацами
\linespread{1.3} % Полуторный интвервал (ГОСТ Р 7.0.11-2011, 5.3.6)

% Оформление списков
\setlist{leftmargin=1.5cm,topsep=0pt}
% Используем дефис для ненумерованных списков (ГОСТ 2.105-95, 4.1.7)
\renewcommand{\labelitemi}{\normalfont\bfseries{--}}

%%% Цвета %%%
% Цвета для кода
\definecolor{string}{HTML}{B40000} % цвет строк в коде
\definecolor{comment}{HTML}{008000} % цвет комментариев в коде
\definecolor{keyword}{HTML}{1A00FF} % цвет ключевых слов в коде
\definecolor{morecomment}{HTML}{8000FF} % цвет include и других элементов в коде
\definecolor{сaptiontext}{HTML}{FFFFFF} % цвет текста заголовка в коде
\definecolor{сaptionbk}{HTML}{999999} % цвет фона заголовка в коде
\definecolor{bk}{HTML}{FFFFFF} % цвет фона в коде
\definecolor{frame}{HTML}{999999} % цвет рамки в коде
\definecolor{brackets}{HTML}{B40000} % цвет скобок в коде
% Цвета для гиперссылок
\definecolor{linkcolor}{HTML}{799B03} % цвет ссылок
\definecolor{urlcolor}{HTML}{799B03} % цвет гиперссылок
\definecolor{citecolor}{HTML}{799B03} % цвет гиперссылок
\definecolor{gray}{rgb}{0.4,0.4,0.4}
\definecolor{tableheadcolor}{HTML}{E5E5E5} % цвет шапки в таблицах
\definecolor{darkblue}{rgb}{0.0,0.0,0.6}
% Цвета для графиков
\definecolor{plotcoordinate}{HTML}{88969C}% цвет точек на координатых осях (минимум и максимум)
\definecolor{plotgrid}{HTML}{ECECEC} % цвет сетки
\definecolor{plotmain}{HTML}{97BBCD} % цвет основного графика
\definecolor{plotsecond}{HTML}{FF0000} % цвет второго графика, если графика только два
\definecolor{plotsecondgray}{HTML}{CCCCCC} % цвет второго графика, если графика только два. В сером виде.
\definecolor{darkgreen}{HTML}{799B03} % цвет темно-зеленого

%%% Отображение кода %%%
% Настройки отображения кода
\lstset{
language=C++, % Язык кода по умолчанию
morekeywords={*,...}, % если хотите добавить ключевые слова, то добавляйте
% Цвета
keywordstyle=\color{keyword}\ttfamily\bfseries,
%stringstyle=\color{string}\ttfamily,
stringstyle=\ttfamily\color{red!50!brown},
commentstyle=\color{comment}\ttfamily\itshape,
morecomment=[l][\color{morecomment}]{\#}, 
% Настройки отображения     
breaklines=true, % Перенос длинных строк
basicstyle=\ttfamily\footnotesize, % Шрифт для отображения кода
backgroundcolor=\color{bk}, % Цвет фона кода
frame=lrb,xleftmargin=\fboxsep,xrightmargin=-\fboxsep, % Рамка, подогнанная к заголовку
rulecolor=\color{frame}, % Цвет рамки
tabsize=3, % Размер табуляции в пробелах
% Настройка отображения номеров строк. Если не нужно, то удалите весь блок
%numbers=left, % Слева отображаются номера строк
%stepnumber=1, % Каждую строку нумеровать
%numbersep=5pt, % Отступ от кода 
%numberstyle=\small\color{black}, % Стиль написания номеров строк
% Для отображения русского языка
extendedchars=true,
literate={Ö}{{\"O}}1
  {Ä}{{\"A}}1
  {Ü}{{\"U}}1
  {ß}{{\ss}}1
  {ü}{{\"u}}1
  {ä}{{\"a}}1
  {ö}{{\"o}}1
  {~}{{\textasciitilde}}1
  {а}{{\selectfont\char224}}1
  {б}{{\selectfont\char225}}1
  {в}{{\selectfont\char226}}1
  {г}{{\selectfont\char227}}1
  {д}{{\selectfont\char228}}1
  {е}{{\selectfont\char229}}1
  {ё}{{\"e}}1
  {ж}{{\selectfont\char230}}1
  {з}{{\selectfont\char231}}1
  {и}{{\selectfont\char232}}1
  {й}{{\selectfont\char233}}1
  {к}{{\selectfont\char234}}1
  {л}{{\selectfont\char235}}1
  {м}{{\selectfont\char236}}1
  {н}{{\selectfont\char237}}1
  {о}{{\selectfont\char238}}1
  {п}{{\selectfont\char239}}1
  {р}{{\selectfont\char240}}1
  {с}{{\selectfont\char241}}1
  {т}{{\selectfont\char242}}1
  {у}{{\selectfont\char243}}1
  {ф}{{\selectfont\char244}}1
  {х}{{\selectfont\char245}}1
  {ц}{{\selectfont\char246}}1
  {ч}{{\selectfont\char247}}1
  {ш}{{\selectfont\char248}}1
  {щ}{{\selectfont\char249}}1
  {ъ}{{\selectfont\char250}}1
  {ы}{{\selectfont\char251}}1
  {ь}{{\selectfont\char252}}1
  {э}{{\selectfont\char253}}1
  {ю}{{\selectfont\char254}}1
  {я}{{\selectfont\char255}}1
  {А}{{\selectfont\char192}}1
  {Б}{{\selectfont\char193}}1
  {В}{{\selectfont\char194}}1
  {Г}{{\selectfont\char195}}1
  {Д}{{\selectfont\char196}}1
  {Е}{{\selectfont\char197}}1
  {Ё}{{\"E}}1
  {Ж}{{\selectfont\char198}}1
  {З}{{\selectfont\char199}}1
  {И}{{\selectfont\char200}}1
  {Й}{{\selectfont\char201}}1
  {К}{{\selectfont\char202}}1
  {Л}{{\selectfont\char203}}1
  {М}{{\selectfont\char204}}1
  {Н}{{\selectfont\char205}}1
  {О}{{\selectfont\char206}}1
  {П}{{\selectfont\char207}}1
  {Р}{{\selectfont\char208}}1
  {С}{{\selectfont\char209}}1
  {Т}{{\selectfont\char210}}1
  {У}{{\selectfont\char211}}1
  {Ф}{{\selectfont\char212}}1
  {Х}{{\selectfont\char213}}1
  {Ц}{{\selectfont\char214}}1
  {Ч}{{\selectfont\char215}}1
  {Ш}{{\selectfont\char216}}1
  {Щ}{{\selectfont\char217}}1
  {Ъ}{{\selectfont\char218}}1
  {Ы}{{\selectfont\char219}}1
  {Ь}{{\selectfont\char220}}1
  {Э}{{\selectfont\char221}}1
  {Ю}{{\selectfont\char222}}1
  {Я}{{\selectfont\char223}}1
  {і}{{\selectfont\char105}}1
  {ї}{{\selectfont\char168}}1
  {є}{{\selectfont\char185}}1
  {ґ}{{\selectfont\char160}}1
  {І}{{\selectfont\char73}}1
  {Ї}{{\selectfont\char136}}1
  {Є}{{\selectfont\char153}}1
  {Ґ}{{\selectfont\char128}}1
  {\{}{{{\color{brackets}\{}}}1 % Цвет скобок {
  {\}}{{{\color{brackets}\}}}}1 % Цвет скобок }
}
% Для настройки заголовка кода
\DeclareCaptionFont{white}{\color{сaptiontext}}
\DeclareCaptionFormat{listing}{\parbox{\linewidth}{\colorbox{сaptionbk}{\parbox{\linewidth}{#1#2#3}}\vskip-4pt}}
\captionsetup[lstlisting]{format=listing,labelfont=white,textfont=white}
\renewcommand{\lstlistingname}{Код} % Переименование Listings в нужное именование структуры
% Для отображения кода формата xml
\lstdefinelanguage{XML}
{
  morestring=[s]{"}{"},
  morecomment=[s]{?}{?},
  morecomment=[s]{!--}{--},
  commentstyle=\color{comment},
  moredelim=[s][\color{black}]{>}{<},
  moredelim=[s][\color{red}]{\ }{=},
  stringstyle=\color{string},
  identifierstyle=\color{keyword}
}

%%% Гиперссылки %%%
\hypersetup{pdfstartview=FitH,  linkcolor=linkcolor,urlcolor=urlcolor,citecolor=citecolor, colorlinks=true}

%%% Псевдокоды %%%
% Добавляем свои блоки
\makeatletter
\algblock[ALGORITHMBLOCK]{BeginAlgorithm}{EndAlgorithm}
\algblock[BLOCK]{BeginBlock}{EndBlock}
\makeatother

% Нумерация блоков
\usepackage{caption}% http://ctan.org/pkg/caption
\captionsetup[ruled]{labelsep=period}
\makeatletter
\@addtoreset{algorithm}{chapter}% algorithm counter resets every chapter
\makeatother
\renewcommand{\thealgorithm}{\thechapter.\arabic{algorithm}}% Algorithm # is <chapter>.<algorithm>

%%% Формулы %%%
%Дублирование символа при переносе
\newcommand{\hmm}[1]{#1\nobreak\discretionary{}{\hbox{\ensuremath{#1}}}{}}

%%% Таблицы %%%
% Раздвигаем в таблице без границ отступы между строками в новой команде
\newenvironment{tabularwide}%
{\setlength{\extrarowheight}{0.3cm}\begin{tabular}{  p{\dimexpr 0.45\linewidth-2\tabcolsep} p{\dimexpr 0.55\linewidth-2\tabcolsep}  }}  {\end{tabular}}
\newenvironment{tabularwide08}%
{\setlength{\extrarowheight}{0.3cm}\begin{tabular}{  p{\dimexpr 0.8\linewidth-2\tabcolsep} p{\dimexpr 0.2\linewidth-2\tabcolsep}  }}  {\end{tabular}}

% Многострочная ячейка в таблице
\newcommand{\specialcell}[2][c]{%
  {\renewcommand{\arraystretch}{1}\begin{tabular}[t]{@{}l@{}}#2\end{tabular}}}

% Многострочная ячейка, где текст не может выйти за границы
\newcolumntype{P}[1]{>{\raggedright\arraybackslash}p{#1}}
\newcommand{\specialcelltwoin}[2][c]{%
  {\renewcommand{\arraystretch}{1}\begin{tabular}[t]{@{}P{1.98in}@{}}#2\end{tabular}}}
  
% Команда для переворачивания текста в ячейке таблицы на 90 градусов
\newcommand*\rot{\rotatebox{90}}

%%% Рисование графиков %%%
\pgfplotsset{
every axis legend/.append style={at={(0.5,-0.13)},anchor=north,legend cell align=left},
tick label style={font=\tiny\scriptsize},
label style={font=\scriptsize},
legend style={font=\scriptsize},
grid=both,
minor tick num=2,
major grid style={plotgrid},
minor grid style={plotgrid},
axis lines=left,
legend style={draw=none},
/pgf/number format/.cd,
1000 sep={}
}
% Карта цвета для трехмерных графиков в стиле графиков Mathcad
\pgfplotsset{
/pgfplots/colormap={mathcad}{rgb255(0cm)=(76,0,128) rgb255(2cm)=(0,14,147) rgb255(4cm)=(0,173,171) rgb255(6cm)=(32,205,0) rgb255(8cm)=(229,222,0) rgb255(10cm)=(255,13,0)}
}
% Карта цвета для трехмерных графиков в стиле графиков Matlab
\pgfplotsset{
/pgfplots/colormap={matlab}{rgb255(0cm)=(0,0,128) rgb255(1cm)=(0,0,255) rgb255(3cm)=(0,255,255) rgb255(5cm)=(255,255,0) rgb255(7cm)=(255,0,0) rgb255(8cm)=(128,0,0)}
}

%%% Разное %%%
% Галочки для отмечания в тескте вариантов как OK
\def\checkmark{\tikz\fill[black,scale=0.3](0,.35) -- (.25,0) -- (1,.7) -- (.25,.15) -- cycle;}
\def\checkmarkgreen{\tikz\fill[darkgreen,scale=0.3](0,.35) -- (.25,0) -- (1,.7) -- (.25,.15) -- cycle;} 
\def\checkmarkred{\tikz\fill[red,scale=0.3](0,.35) -- (.25,0) -- (1,.7) -- (.25,.15) -- cycle;}
\def\checkmarkbig{\tikz\fill[black,scale=0.5](0,.35) -- (.25,0) -- (1,.7) -- (.25,.15) -- cycle;}
\def\checkmarkbiggreen{\tikz\fill[darkgreen,scale=0.5](0,.35) -- (.25,0) -- (1,.7) -- (.25,.15) -- cycle;} 
\def\checkmarkbigred{\tikz\fill[red,scale=0.5](0,.35) -- (.25,0) -- (1,.7) -- (.25,.15) -- cycle;}

%% Следующие блоки расскоментировать при необходимости

%%% Кодировки и шрифты %%%
%\ifxetex
%\setmainlanguage{russian}
%\setotherlanguage{english}
%\defaultfontfeatures{Ligatures=TeX,Mapping=tex-text}
%\setmainfont{Times New Roman}
%\newfontfamily\cyrillicfont{Times New Roman}
%\setsansfont{Arial}
%\newfontfamily\cyrillicfontsf{Arial}
%\setmonofont{Courier New}
%\newfontfamily\cyrillicfonttt{Courier New}
%\else
%\IfFileExists{pscyr.sty}{\renewcommand{\rmdefault}{ftm}}{}
%\fi

%%% Колонтитулы %%%
% Порядковый номер страницы печатают на середине верхнего поля страницы (ГОСТ Р 7.0.11-2011, 5.3.8)
%\makeatletter
%\let\ps@plain\ps@fancy              % Подчиняем первые страницы каждой главы общим правилам
%\makeatother
%\pagestyle{fancy}                   % Меняем стиль оформления страниц
%\fancyhf{}                          % Очищаем текущие значения
%\fancyhead[C]{\thepage}             % Печатаем номер страницы на середине верхнего поля
%\renewcommand{\headrulewidth}{0pt}  % Убираем разделительную линию

%%% Оглавление %%%
%\renewcommand{\cftchapdotsep}{\cftdotsep}
%\renewcommand{\cftchapleader}{\cftdotfill{\cftdotsep}}
%\renewcommand{\cftsecleader}{\cftdotfill{\cftdotsep}}
%\renewcommand{\cftfigleader}{\cftdotfill{\cftdotsep}}
%\renewcommand{\cfttableader}{\cftdotfill{\cftdotsep}} %Подключаем модуль стилей

\begin{document}

%%% HarrixLaTeXDocumentTemplate
%%% Версия 1.22
%%% Шаблон документов в LaTeX на русском языке. Данный шаблон применяется в проектах HarrixTestFunctions, MathHarrixLibrary, Standard-Genetic-Algorithm  и др.
%%% https://github.com/Harrix/HarrixLaTeXDocumentTemplate
%%% Шаблон распространяется по лицензии Apache License, Version 2.0.

%%% Именования %%%
\renewcommand{\abstractname}{Аннотация}
\renewcommand{\alsoname}{см. также}
\renewcommand{\appendixname}{Приложение} % (ГОСТ Р 7.0.11-2011, 5.7)
\renewcommand{\bibname}{Список литературы} % (ГОСТ Р 7.0.11-2011, 4)
\renewcommand{\ccname}{исх.}
\renewcommand{\chaptername}{Глава}
\renewcommand{\contentsname}{Оглавление} % (ГОСТ Р 7.0.11-2011, 4)
\renewcommand{\enclname}{вкл.}
\renewcommand{\figurename}{Рисунок} % (ГОСТ Р 7.0.11-2011, 5.3.9)
\renewcommand{\headtoname}{вх.}
\renewcommand{\indexname}{Предметный указатель}
\renewcommand{\listfigurename}{Список рисунков}
\renewcommand{\listtablename}{Список таблиц}
\renewcommand{\pagename}{Стр.}
\renewcommand{\partname}{Часть}
\renewcommand{\refname}{Список литературы} % (ГОСТ Р 7.0.11-2011, 4)
\renewcommand{\seename}{см.}
\renewcommand{\tablename}{Таблица} % (ГОСТ Р 7.0.11-2011, 5.3.10)

%%% Псевдокоды %%%
% Перевод данных об алгоритмах
\renewcommand{\listalgorithmname}{Список алгоритмов}
\floatname{algorithm}{Алгоритм}

% Перевод команд псевдокода
\algrenewcommand\algorithmicwhile{\textbf{До тех пока}}
\algrenewcommand\algorithmicdo{\textbf{выполнять}}
\algrenewcommand\algorithmicrepeat{\textbf{Повторять}}
\algrenewcommand\algorithmicuntil{\textbf{Пока выполняется}}
\algrenewcommand\algorithmicend{\textbf{Конец}}
\algrenewcommand\algorithmicif{\textbf{Если}}
\algrenewcommand\algorithmicelse{\textbf{иначе}}
\algrenewcommand\algorithmicthen{\textbf{тогда}}
\algrenewcommand\algorithmicfor{\textbf{Цикл. }}
\algrenewcommand\algorithmicforall{\textbf{Выполнить для всех}}
\algrenewcommand\algorithmicfunction{\textbf{Функция}}
\algrenewcommand\algorithmicprocedure{\textbf{Процедура}}
\algrenewcommand\algorithmicloop{\textbf{Зациклить}}
\algrenewcommand\algorithmicrequire{\textbf{Условия:}}
\algrenewcommand\algorithmicensure{\textbf{Обеспечивающие условия:}}
\algrenewcommand\algorithmicreturn{\textbf{Возвратить}}
\algrenewtext{EndWhile}{\textbf{Конец цикла}}
\algrenewtext{EndLoop}{\textbf{Конец зацикливания}}
\algrenewtext{EndFor}{\textbf{Конец цикла}}
\algrenewtext{EndFunction}{\textbf{Конец функции}}
\algrenewtext{EndProcedure}{\textbf{Конец процедуры}}
\algrenewtext{EndIf}{\textbf{Конец условия}}
\algrenewtext{EndFor}{\textbf{Конец цикла}}
\algrenewtext{BeginAlgorithm}{\textbf{Начало алгоритма}}
\algrenewtext{EndAlgorithm}{\textbf{Конец алгоритма}}
\algrenewtext{BeginBlock}{\textbf{Начало блока. }}
\algrenewtext{EndBlock}{\textbf{Конец блока}}
\algrenewtext{ElsIf}{\textbf{иначе если }} %Подключаем модуль переименования некоторых команд

%%%%%%%%%%%%%%%%%%%%%%%%%%%%%%%%%%%%%%%%%%%%%%%%%%%
% % % % % % % % Титульная страница % % % % % % % % 
%%%%%%%%%%%%%%%%%%%%%%%%%%%%%%%%%%%%%%%%%%%%%%%%%%%
\thispagestyle{empty}

\begin{center}
\MakeUppercase{МФТИ ФИВТ} \par
\MakeUppercase{Московский физико-технический институт (государственный университет), факультет инноваций и высоких технологий} \par 
\par
\end{center}

\vspace{50mm}

\begin{center}
{\large Луничкин Егор Валериевич}
\end{center}

\vspace{5mm}
\begin{center}
{\bf \large \MakeUppercase{Теорема Блума об ускорениях}
\par}

\vspace{20mm}

\end{center}

\vspace{100mm}


\begin{center}
{Долгопрудный -- 2015}
\end{center}

%%%%%%%%%%%%%%%%%%%%%%%%%%%%%%%%%%%%%%%%%%%%%%%%%%%
% % % % % % % % Содержание % % % % % % % % 
%%%%%%%%%%%%%%%%%%%%%%%%%%%%%%%%%%%%%%%%%%%%%%%%%%%
\tableofcontents
\clearpage

%%%%%%%%%%%%%%%%%%%%%%%%%%%%%%%%%%%%%%%%%%%%%%%%%%%
% % % % % % % % Глава 1 % % % % % % % % 
%%%%%%%%%%%%%%%%%%%%%%%%%%%%%%%%%%%%%%%%%%%%%%%%%%%
\chapter{Введение}
\section{Аннотация}
Теорема Блума об ускорениях~--- это одна из базовых теорем в теории сложности. В этой работе изложено одно из упрощённых доказательств этой теоремы, которое, однако, хорошо обобщается на весь класс общерекурсивных (а следовательно, и частично рекурсивных) функций. 
\section{Обозначения}
В этой статье используется нотация, аналогичная \cite{Rogers}. $\lambda i \phi_i$~--- стандартная индексация частично рекурсивных функций. $\mathbb{N}$~--- множество всех неотрицательных целых. $\lambda i D_i$~--- все конечные подмножества $\mathbb{N}$. Аналогично, $\lambda i F_i$~--- индексация всех финитных функций, определённых на участке $\{0, 1, 2, \dots , n\}$. $\lambda_i \Phi_i$~--- любая мера вычислительной или ресурсной сложности задачи по Блуму. В частности, для всех $i$, область определения $\Phi_i$ в точности совпадает с областью определения $\phi_i$, и отношение $\Phi_i \leqq y$ рекурсивно разрешимо. Простым языком, $\Phi_i (x)$~--- количество времени или памяти, которое потребуется машине Тьюринга $i$ на входе $x$. Если $f$~--- некоторая функция, а  $S$~--- множество, то  $f/S$~--- ограничение $f$ на $S$.

%%%%%%%%%%%%%%%%%%%%%%%%%%%%%%%%%%%%%%%%%%%%%%%%%%%
% % % % % % % % Глава 2 % % % % % % % % 
%%%%%%%%%%%%%%%%%%%%%%%%%%%%%%%%%%%%%%%%%%%%%%%%%%%

\chapter{Теорема и её доказательство}

Доказательство, которое здесь будет приведено, является по своей сути упрощением доказательства Блума из \cite{Blum}. Хорошо известно, что для всех мер, связанных рекурсивно \cite{Blum}, следует, что если доказательство можно провести для одной из них, то оно будет верно для всех остальных, поэтому не будем проводить доказательство в самом общем случае. Начнём с того предположения, что для любой меры верно:

\begin{equation} \label{eq:2.1}
\Phi_{S(i,x)} (y) \leqq \Phi_i (x,y)
\end{equation}
и если $x\notin$ области определения $F_v$, 
\begin{equation} \label{eq:2.2}
\Phi_{\rho (i,v)} (x)\leqq \Phi_i (x) 
\end{equation}

Например, лента машины Тьюринга удовлетворяет этим условиям. Также предположим, что можно определить функцию рекурсивно не только по её предыдущим значениям, но и по предыдущим запускам. Это можно понять из следующего рассуждения: если мы использовали программу для вычисления значения функции на каком-либо начальном аргументе, мы можем узнать вычислительные ресурсы, которые потребовались для этого, даже не зная явно программу, которая вычисляет эту функцию. Теперь мы хотим доказать:

\section{Формулировка}

\textbf{Теорема} (Теорема Блума об ускорениях \cite{Blum}). \textit{Для каждой общерекурсивной функции $r(x,y)$ мы можем найти общерекурсивную функцию $f(x)$ такую, что из $\phi_i = f$ следует, что $\exists j$ такое, что $\phi_j = f$ и $\Phi_i (x) > r(x, \Phi_j (x))$ во всех точках кроме, быть может, конечного их числа.}

\section{Доказательство}

\subsection{Идея доказательства}

\underline{Доказательство.} Идея доказательства в том, чтобы построить функцию $\phi_t(u, x)$ такую, чтобы для любого $u$ было верно $\lambda x \phi_t (u, x) = \lambda x\phi_t(0, x)$. Положим $f=\lambda x \phi_t(0, x)$, и при построении будем использовать первый параметр $u$ в диагонализации по всем возможным программам, чтобы гарантировать, что если $\phi_i = f$, то $\Phi_i(x) \geqq r(x, \Phi_t(i+1, x))$ во всех точках, кроме, быть может, конечного их числа. В силу \eqref{eq:2.1} этого будет достаточно для доказательства теоремы. К сожалению, может быть доказано \cite{Blum}, что в общем случае невозможно сделать это, сохраняя $\lambda x \phi_t(u, x) = \lambda x\phi_t(0, x) \ \forall u$. Однако с учётом \eqref{eq:2.2} будет достаточно показать, что $\lambda x \phi_t(u, x) = \lambda x\phi_t (0, x)$ для всех точек, кроме конечного их числа и $\forall u$, так как всегда можно подправить $\lambda x \phi_t(u, x)$ на конечном числе входов, не увеличивая основной оценки сложности. Предлагается для начала доказать ослабленную версию \textbf{теоремы}:

\subsection{Лемма}

\textbf{Лемма} (Эффективные псевдоускорения; Блум \cite{Blum2}). \textit {Пусть $\Phi$~--- мера, удовлетворяющая \eqref{eq:2.1}. Тогда для каждой общерекурсивной функции $r(x,y)$ мы можем эффективно найти общерекурсивную функцию $f$, такую, что для любого данного $i$, для которого $\phi_i = f$, мы можем эффективно указать $j$, для которых $\phi_j = f$ и $\Phi_i(x) > r(x, \Phi_j(x))$ (для всех функций, кроме, быть может, конечного их числа).}

\underline{Доказательство.} Определим программу $t$, вычисляющую частную рекурсивную функцию двух переменных $\phi_t(u, x)$. Пусть $f = \lambda x \phi_t(0, x)$. Нужно показать, что $\phi_t$~--- общая. $\phi_t(u, x)$ задаётся рекурсивно через $\phi_t(u, 0), \phi_t(u, 1), \cdots , \phi_t(u, x-1)$, а также, если $u < x$, через $\phi_t(x, x), \phi_t(x-1, x), \phi_t(x-2, x), \cdots , \phi_t(u+1, x)$. Конкретно, на уровне $x$ зададим:

\begin{equation}
C_{u,x} = \left\{ i | u \leqq i < x \mbox{ и } i \notin \bigcup_{y<x} C_{u,v} \mbox{ и } \Phi_i(x) \leqq r(x, \phi_t(i+1, x)) \right\}
\end{equation}

Будем говорить, что $C_{u,x}$~--- это набор программ, \textit{отменённых} на этапе $x$ при вычислении $\phi_t(u, x)$. Тогда определим:

\begin{equation}
\phi_t(u,x) = 1 + \max\{\phi_i(x) | i \in C_{u,x}\}
\end{equation}

То есть, $\phi_i(x) = 1 + \max\{u \leqq i < x \mbox{ и } \Phi_i(x) \leqq r(x, \Phi_t(i+1, x))$, и $i$ не была отменена раньше при вычислении $\lambda x \phi_t(u, x)\}$.

Это прямо следует из определения, что $\phi_t(u,0) = 1$ для всех $u$ и $\phi_t(u, x) = 1$ всякий раз, когда $u \geqq x$. Более того, для любого $u < x$, для того, чтобы определить $\phi_t(u, x)$, необходимо определить $\Phi_t(x, x), \Phi_t(x-1, x), \Phi_t(x-2, x), \cdots , \Phi_t(u+1, x)$ и $\phi_t(u, 0), \phi_t(u, 1), \phi_t(u, 2), \cdots , \phi_t(u,x-1)$. Отсюда видно, что достаточно определить $\phi_t(x, x), \phi_t(x-1, x), \cdots , \phi_t(u+1, x) ; \phi_t(u, 0), \phi_t(u, 1), \cdots , \phi_t(u, x-1)$.

Продолжая по индукции, предположим, что $\lambda u \phi_t(u, x^{'})$~--- общая функция для всех $x^{'} < x$. Можно заметить, что $\phi_t(u, x) = 1$ для всех $u \geqq x$, значит, $\phi_t(x-1, x)$ определена. Аналогично, $\phi_t(x-2, x)$ определена. Продолжая рекурсивно до $\phi_t(0, x)$, увидим, что $\lambda u \phi_t(u, x)$~--- общая. Таким образом, по индукции доказано, что $\phi_t$~--- общая функция.

Очевидно, что $C_{0,x} - \{0, 1, \dots , u-1\} = C_{u,x}$. Более того, очевидно, что для каждого $u$ существует $n_u$, такой, что если $i < u$ и $i \in \bigcup_y C_{0,y}$, то

\begin{equation}
i \in \bigcup_{y \le n_u} C_{0,y}
\end{equation}

Таким образом, никакое $i < u$ не принадлежит $C_{0,x}$ для $x > n_u$. Отсюда сразу получаем, что для $x > n_u \ C_{0,x} = C_{u,x}$, а значит, $\phi_t(0, x) = \phi_t(u, x)$ для $x > n_u$.

Окончательно, если $\phi_i = \lambda x \phi(0, x)$, должно выполняться:

\begin{equation} \label{eq:2.6}
\phi_i(x) > r(x, \Phi_t(i+1, x)) \mbox{ для всех } x > i
\end{equation}

Иначе, при вычислении $\lambda x \phi_t(0,x)$, мы должны \textit{отменить} $i$ для первого такого $x > i$, приходя к противоречию $\phi_t(0,x) \ne \phi_i(x)$. Таким образом, доказательство фактически завершено.

Мы можем предположить без потери общности, что $r$ монотонна по своему второму аргументу. Тогда из \eqref{eq:2.6} и \eqref{eq:2.1} имеем:

\begin{equation} \label{eq:2.7}
\Phi_i(x) > r(x, \Phi_t(i+1, x)) \geqq r(x, \Phi_{S(t, i+1)}(x))
\end{equation}
для всех точек кроме, быть может, конечного числа.

Таким образом, мы завершили доказательство \textbf{леммы}.

\subsection{Завершение доказательства}

Так как для каждого $i$, $\lambda x \phi_t(0, x) = \phi_{\rho(S(t, i+1), v)}(x)$, то для некоторых $v$ в условиях, удовлетворяющих \eqref{eq:2.2}, из \eqref{eq:2.7} мы получаем:

\begin{equation} \label{eq:2.8}
\Phi_i(x) \geqq r(x, \Phi_{\rho(S(t, i+1), v)}(x)
\end{equation}
для всех точек кроме, быть может, конечного числа.

Таким образом, \textbf{теорема} доказана при данных условиях.

Как было отмечено ранее, доказательство \textbf{теоремы} при произвольных условиях не привязано к рекурсивному отношению, но прямое доказательство при произвольных условиях может быть получено изменением приведённого выше доказательства. Чтобы его провести, мы просто определим $\phi_t$ как приложение рекурсивной теоремы, изменяя определение $C_{u,x}$ на:

\begin{equation}
C_{u,x} = \left\{ i | u \leqq i < x \mbox{ и } i \notin \bigcup_{y < x} C_{u,y} \mbox{ и } \Phi_i(x) \leqq \max_{r<x} r(x, \Phi_{\rho(S(t, i+1), v)}(x) \right\}
\end{equation}

Теперь предыдущее доказательство повторяется практически дословно, исключая тот факт, что доказательство \eqref{eq:2.6} прямо переводится в доказательство желаемого, а именно \eqref{eq:2.8}.

На самом деле, \textbf{теорема об ускорениях} из \cite{Blum} немного сложнее, чем доказанная \textbf{теорема}, потому что в \cite{Blum} $f$ берётся как функция, принимающая значения $0-1$. Однако переход к случаю функции, принимающей любые значения, не является целью этой работы и не представляет большого интереса. 

%%%%%%%%%%%%%%%%%%%%%%%%%%%%%%%%%%%%%%%%%%%%%%%%%%%
% % % % % % % % Список литературы % % % % % % % % 
%%%%%%%%%%%%%%%%%%%%%%%%%%%%%%%%%%%%%%%%%%%%%%%%%%%

\begin{thebibliography}{9}

\bibitem{Rogers}
Hartley Rogers, Jr., \textit{Theory of recursive functions and effective computability},
McGraw-Hill, New York, 1967. MR \textbf{37} \#61.

\bibitem{Blum}
M. Blum, \textit{A machine-independent theory of the complexity of recursive functions},
J. Assoc. Comput. Mach. \textbf{14} (1967), 322---336. MR \textbf{38} \#4213.

\bibitem{Blum2}
M. Blum, \textit{On effective procedures for sppeding up algorithms},
J. Assoc. Comput. Mach. \textbf{18} (1971), 290---305.

\end{thebibliography}
\end{document}
